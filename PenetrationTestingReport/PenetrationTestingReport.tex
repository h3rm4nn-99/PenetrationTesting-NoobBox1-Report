\documentclass[a4paper, 12pt, oneside]{article}
\usepackage[utf8]{inputenc}
\usepackage[margin=3cm, bindingoffset=1cm]{geometry}
\linespread{1.5}
\usepackage{float}
\usepackage{csquotes}
\usepackage{subfig}
\usepackage{graphicx}
\usepackage{indentfirst}
\usepackage{fancyhdr}
\usepackage{alphabeta}
\usepackage{algpseudocode}
\usepackage{algorithm}
\usepackage{hyperref}
\usepackage[T1]{fontenc}
\usepackage{listings}
\usepackage[htt]{hyphenat}
\usepackage{pgfplots}

\usepackage[
    backend=biber,
    sorting=none
]{biblatex}
\addbibresource{bibliography.bib}


\setlength{\parindent}{1cm}

\pagestyle{fancy}
\fancyhf{}
\fancyhead[C]{\textbf{\leftmark}}
\fancyfoot[C]{\thepage}
\renewcommand{\headrulewidth}{1pt}
\renewcommand{\footrulewidth}{1pt}
\renewcommand{\contentsname}{Indice}

\usepackage[Conny]{fncychap}

  
\begin{document}
\begin{titlepage}
    \begin{center}
        \LARGE{\uppercase{Università degli Studi di Salerno}}\\
        \vspace{5mm}
    	\uppercase{\normalsize Dipartimento di Informatica }\\
    \end{center}
    \begin{figure}[H]
        \centering
        \includegraphics[width=0.35\textwidth]{logo_unisa}
    \end{figure}
    
    \begin{center}
        \normalsize{Corso di \textbf{Penetration Testing and Ethical Hacking}}\\
    	\vspace{10mm}
    	\LARGE{\textbf{\textsc{NoobBox-1}:\\ Penetration Testing Report}}\\
    	\vspace{3mm}
        \large{\uppercase{Anno Accademico 2022/2023}}
    \end{center}

    \vspace{70mm}
    \noindent
    \begin{minipage}[t]{0.6\textwidth}
    	Docente:\\\textbf{Prof. Arcangelo Castiglione}
    	\vspace{10mm}\\
    \end{minipage}
    \hfill
    \begin{minipage}[t]{0.4\textwidth}\raggedleft
    	Studente: \\\textbf{Hermann Senatore}
    \end{minipage}
\end{titlepage}

\tableofcontents
\newpage

\section{Scopi e struttura del documento}

Il \textbf{Penetration Testing Report} consiste in un resoconto, articolato in diversi livelli di dettaglio, sulle varie fasi del processo di Penetration Testing condotto sull'asset.

Tale documento è articolato in diverse sezioni. Ciascuna di queste si concentra su aspetti diversi. In particolare:

\begin{enumerate}
    \item \textbf{Executive Summary}: in questa sezione viene svolta una sintesi dei risultati del processo e dello stato generale di sicurezza del sistema analizzato;
    \item \textbf{Engagement Highlights}: in questa sezione vengono esplicitate le regole di ingaggio tra chi ha commissionato l'indagine ed il Penetration Tester, le metodologie e gli obiettivi dell'analisi;
    \item \textbf{Vulnerability Report}: in questa sezione viene fornita una visione d'insieme delle problematiche di sicurezza di cui l'asset è affetto;
    \item \textbf{Findings Summary}: in questa sezione vengono presentate con maggior livello di dettaglio le vulnerabilità riscontrate durante il processo di Penetration Testing;
    \item \textbf{Remediation Report}: in questa sezione sono proposte eventuali soluzioni alle vulnerabilità ed alle debolezze descritte nella sezione precedente;
    \item \textbf{Detailed Summary}: in questa sezione è presente una discussione approfondita sulle problematiche individuate in precedenza.
\end{enumerate}

\newpage
\section{Executive Summary}
Il processo di Penetration Testing che questo documento sommarizza è stato svolto su un asset \textit{vulnerable by default} denominato \textbf{NoobBox-1} reperibile presso la piattaforma \textbf{VulnHub} all'indirizzo \url{https://www.vulnhub.com/entry/noobbox-1,664/}. Nato come sfida CTF, è stato utilizzato dall'autore del presente documento per prendere confidenza con i tool e le metodologie più utilizzate nel contesto del Penetration Testing.

Gli obiettivi che sono stati fissati e raggiunti in questo processo consistono sostanzialmente in:

\begin{itemize}
    \item Identificazione ed enumerazione completa dei servizi presenti all'interno dell'asset;
    \item Rilevamento delle vulnerabilità e delle debolezze presenti all'interno dell'asset;
    \item Ottenere accesso privilegiato alla macchina;
    \item Provvedere all'installazione di software specializzato per permettere l'accesso persistente all'asset.
\end{itemize}

L'attività di Penetration Testing è stata condotta a partire dal giorno 22 maggio 2023 ed ha avuto una connotazione \textbf{grey box} poiché alcune informazioni sono reperibili direttamente dalla piattaforma VulnHub e che fungono da punto di partenza.

Il livello di rischio derivato dall'analisi è stato classificato come \textbf{Medio-Alto} perché sebbene non siano presenti vulnerabilità intrinseche dei servizi in esecuzione che siano direttamente sfruttabili, è stato comunque possibile ottenere accesso privilegiato alla macchina mediante \textbf{errori di configurazione} e pratiche di sicurezza \textbf{scorrette}.

Una volta applicate le migliorie e le correzioni descritte nella sezione denominata \textbf{Remediation Report}, il livello di rischio scenderebbe ad un livello tale da non permettere più l'accesso non autorizzato alla macchina.

\newpage

\section{Engagements Highlights}
Poiché l'analisi che questo documento descrive consiste in un progetto di stampo accademico e l'asset considerato ha questo tipo di analisi come scopo dichiarato, non sono state definite delle limitazioni dal punto di vista legale o contrattuale. In particolare, non sono presenti parti dell'asset che non sarebbero dovute essere analizzate così come non è stata imposta alcuna limitazione riguardo gli strumenti da utilizzare. Inoltre, non è stato (ovviamente) previsto alcun accordo di non-divulgazione.

\newpage

\section{Vulnerability Report}
In questa sezione viene proposta un'\textit{overview} delle problematiche di sicurezza presenti sull'asset.

In particolare, le problematiche appartengono alle categorie di:

\begin{itemize}
    \item Information disclosure;
    \item Obsolescenza del software utilizzato dall'asset;
    \item Errori di configurazione e cattive pratiche di sicurezza;
\end{itemize}

\subsection{Information Disclosure}
Alcune informazioni sensibili che possono essere ricollegate a meccanismi di funzionamento interni all'asset sono \textbf{pubblicamente accessibili}. In particolare, visitando il sito web offerto dall'asset all'URL \texttt{/img.jpg} è possibile rinvenire la password dell'utente amministratore di \textbf{Wordpress}. Questo aspetto già da solo consente l'accesso non autorizzato all'asset.

\subsection{Obsolescenza del software utilizzato nell'asset}
Alcune componenti dell'asset sono ormai \textbf{obsolete}. In particolare, il server web \textbf{Apache httpd} è aggiornato alla versione \textbf{2.4.38}, rilasciata il \textbf{22 gennaio 2019}. Questa versione del software è risulta essere afflitta da alcune vulnerabilita. Sebbene nessuna di queste vulnerabilità sia risultata sfruttabile per ottenere l'accesso alla macchina, l'obsolescenza dei software, sia nel caso di \texttt{httpd2} che di qualunque software rende più probabile, a lungo termine, la compromissione dell'asset.

\subsection{Errori di configurazione e cattive pratiche di sicurezza}
L'utente che amministra il sito web creato con \textbf{Wordpress} possiede \textbf{lo stesso username} dell'utente della macchina locale. Una situazione del genere semplifica notevolmente il processo di enumerazione dell'asset, che può condurre alla sua compromissione. A peggiorare la situazione è stata la constatazione di un \textbf{riuso} della password per i due account. Una situazione del genere consente ad un attaccante di accedere ai file personali dell'utente locale dell'asset. 

È stata inoltre rilevata la possibilità per l'utente locale di eseguire software con privilegi \textbf{ingiustamente elevati}. Quest'ultimo aspetto, in caso di accesso non autorizzato, permette di effettuare \textbf{Privilege Escalation}.

\newpage
\printbibliography[title={Riferimenti bibliografici}]
\end{document}