\documentclass[a4paper, 12pt, oneside]{article}
\usepackage[utf8]{inputenc}
\usepackage[margin=3cm, bindingoffset=1cm]{geometry}
\linespread{1.5}
\usepackage{float}
\usepackage{csquotes}
\usepackage{subfig}
\usepackage{graphicx}
\usepackage{indentfirst}
\usepackage{fancyhdr}
\usepackage{alphabeta}
\usepackage{algpseudocode}
\usepackage{algorithm}
\usepackage{hyperref}
\usepackage[T1]{fontenc}
\usepackage{listings}
\usepackage[htt]{hyphenat}
\usepackage{pgfplots}

\usepackage[
    backend=biber,
    sorting=none
]{biblatex}
\addbibresource{bibliography.bib}


\setlength{\parindent}{1cm}

\pagestyle{fancy}
\fancyhf{}
\fancyhead[C]{\textbf{\leftmark}}
\fancyfoot[C]{\thepage}
\renewcommand{\headrulewidth}{1pt}
\renewcommand{\footrulewidth}{1pt}
\renewcommand{\contentsname}{Indice}

\usepackage[Conny]{fncychap}

  
\begin{document}
\begin{titlepage}
    \begin{center}
        \LARGE{\uppercase{Università degli Studi di Salerno}}\\
        \vspace{5mm}
    	\uppercase{\normalsize Dipartimento di Informatica }\\
    \end{center}
    \begin{figure}[H]
        \centering
        \includegraphics[width=0.35\textwidth]{logo_unisa}
    \end{figure}
    
    \begin{center}
        \normalsize{Corso di \textbf{Penetration Testing and Ethical Hacking}}\\
    	\vspace{10mm}
    	\LARGE{\textbf{\textsc{NoobBox-1}:\\ Penetration Testing Report}}\\
    	\vspace{3mm}
        \large{\uppercase{Anno Accademico 2022/2023}}
    \end{center}

    \vspace{70mm}
    \noindent
    \begin{minipage}[t]{0.6\textwidth}
    	Docente:\\\textbf{Prof. Arcangelo Castiglione}
    	\vspace{10mm}\\
    \end{minipage}
    \hfill
    \begin{minipage}[t]{0.4\textwidth}\raggedleft
    	Studente: \\\textbf{Hermann Senatore}
    \end{minipage}
\end{titlepage}

\tableofcontents
\newpage

\section{Scopi e struttura del documento}

Il \textbf{Penetration Testing Report} consiste in un resoconto, articolato in diversi livelli di dettaglio, sulle varie fasi del processo di Penetration Testing condotto sull'asset.

Tale documento è articolato in diverse sezioni. Ciascuna di queste si concentra su aspetti diversi. In particolare:

\begin{enumerate}
    \item \textbf{Executive Summary}: in questa sezione viene svolta una sintesi dei risultati del processo e dello stato generale di sicurezza del sistema analizzato;
    \item \textbf{Engagement Highlights}: in questa sezione vengono esplicitate le regole di ingaggio tra chi ha commissionato l'indagine ed il Penetration Tester, le metodologie e gli obiettivi dell'analisi;
    \item \textbf{Vulnerability Report}: in questa sezione viene fornita una visione d'insieme delle vulnerabilità e delle debolezze individuate durante il processo;
    \item \textbf{Remediation Report}: in questa sezione sono proposte eventuali soluzioni alle vulnerabilità ed alle debolezze descritte nella sezione precedente;
    \item \textbf{Detailed Summary}: in questa sezione è presente una discussione approfondita sulle problematiche individuate in precedenza.
\end{enumerate}

\newpage
\printbibliography[title={Riferimenti bibliografici}]
\end{document}